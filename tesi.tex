\documentclass[corpo=11pt, stile=classica, tipotesi=custom,
greek, evenboxes, english]{toptesi}
\usepackage[utf8]{inputenc}
\usepackage[T1]{fontenc}
\usepackage{lmodern}

\usepackage{hyperref}
\hypersetup{%
	pdfpagemode={UseOutlines},
	bookmarksopen,
	pdfstartview={FitH},
	colorlinks,
	linkcolor={blue},
	citecolor={blue},
	urlcolor={blue}
}

\usepackage{geometry} %for the margins
\newcommand\fillin[1][4cm]{\makebox[#1]{\dotfill}} %for the dotted line in the frontispiace

\usepackage{dcolumn}
\newcolumntype{d}{D{.}{.}{-1} } %to vetical align numbers in tables, along the decimal dot

\usepackage{amsmath}







\usepackage{amsmath, amssymb, amsthm}
\usepackage{enumitem}
\usepackage{tikz}
\usepackage{hyperref}
\usepackage{systeme}
\usepackage{mathtools}
\hypersetup{colorlinks, linkcolor=blue, citecolor=blue}
\usepackage{biblatex}
\addbibresource{bibliografia.bib}


\numberwithin{equation}{chapter}
\newtheorem{teo}{Theorem}[chapter] %in questo modo la numerazione ricomincia da capo ad ogni nuovo capitolo
\newtheorem{defi}[teo]{Definition}
\newtheorem{lem}[teo]{Lemma}
\newtheorem{cor}[teo]{Corollary}
\newtheorem{prop}[teo]{Proposition}
\newtheorem{es}[teo]{Example}

\newcommand{\R}{\mathbb{R}} %scorciatoia per R
\newcommand{\V}{\mathcal{V}} %STFT
\newcommand{\dxdo}{dxd\omega}
\newcommand{\notazione}{\underline{\textbf{Remark Notazionale}}}
\newcommand{\finire}{\fbox{\LARGE DA FINIRE}}

\begin{document}
	
\begin{titlepage}

\newgeometry{left=1cm,right=1cm,top=3cm,bottom=3.5cm}  %specific margins for this page

\begin{center}

{\huge POLITECNICO DI TORINO}\\[1.5cm]
%\textbf{Corso di Laurea\\in Matematica per l'Ingegneria}\\[3cm]
\textbf{Master degree \\in Mathematical Engineering}\\[3cm]

%{\Large Tesi di Laurea}\\[1cm]
{\Large Master's Degree Thesis}\\[0.5cm]
\textbf{\LARGE Recent results on the norm of localization operators}\\[2cm]
\includegraphics[width=0.2\textwidth]{./Pictures/logo_polito_2021.jpg}
\vspace{4cm}


\begin{minipage}{0.85\textwidth}
\begin{flushleft}\large
\textbf{Supervisor} \hfill \textbf{Candidate}\\
Prof. Fabio Nicola \hfill Federico Riccardi\\
%\textit{firma dei relatori} \hfill \textit{firma del candidato}\\[0.35cm]
%\fillin\ \hfill \fillin
\end{flushleft}
\end{minipage}

\vfill

Academic Year 2022-2023
\end{center}

\restoregeometry %restor default margins 

\end{titlepage}

\tableofcontents

\sommario

\chapter{Introduction}

\chapter{Basics of functional analysis}

\chapter{Short-Time Fourier Transform}
\section{STFT}
\subsection{Properties of STFT}
\section{Fock Space and Bargmann Transform}
\section{Faber-Krahn Inequality for the STFT}

Theorem from \cite{nicolatilli_fk}
\begin{teo}\label{faberkrahn}
	For every $f \in L^2(\R^d)$ such that $\|f\|_{L^2} = 1$ and every measurable subset $\Omega \subset \R^{2d}$ with finite measure we have
	\begin{equation*}
		\int_{\Omega}  |\V f(x,\omega)|^2 \dxdo \leq G(|\Omega|)
	\end{equation*}
	where $G(s)$ is given by
	\begin{equation}\label{G}
		G(s) \coloneqq \int_0^s e^{\left(-d!\tau\right)^{1/d}} d\tau
	\end{equation}
\end{teo}


\chapter{Localization Operators}
\section{Definition and properties}
\section{Eigenvalues and eigenfunctions}

\chapter{Recent results from Nicola-Tilli}
\section{Case $q=+\infty$}
\section{Generic case}
Let's now consider the case where both $p$ and $q$ are neither 1 or $+\infty$. The result presented in \cite{nicolatilli_norm} include the case ...
\begin{equation*}
	\| L_F\|_{L_2 \rightarrow L_2} \leq \min\{\kappa_p^{d\kappa_p}A, \, \kappa_q^{d\kappa_q}B\}
\end{equation*}
Suppose that the minimum is given by $\kappa_p^{d\kappa_p}A$, therefore
\begin{equation*}
	\kappa_p^{d\kappa_p}A \leq \kappa_q^{d\kappa_q}B \iff \dfrac{B}{A} \geq \left(\dfrac{\kappa_p^{\kappa_p}}{\kappa_q^{\kappa_q}}\right)^d
\end{equation*}
We can check if the solution of the problem with just the $L^p$ bound solves also the problem with both bounds, that is $F\|_{L^q} \leq B$, where $F$ is given by ...
\begin{align*}
	\| F \|_{L^q}^q &= \int_{\R^{2d}} |F(z)|^q dz = ... = \lambda^q \left(\dfrac{p-1}{q}\right)^d
\end{align*}
Since we want $F$ to satisfy the $L^q$ constraint we should have
\begin{equation*}
	\dfrac{B}{A} \geq \kappa_p^{d\left(\frac1q - \frac1p\right)}\left(\dfrac{p}{q}\right)^{\frac{d}{q}}
\end{equation*}
It would be nice if this bound was less restrictive than the first one. Unfortunately that's not the case, in fact it's always true that
\begin{equation*}
	\left(\dfrac{p'}{q'}\right)^{\frac1{q'}} \left(\dfrac{p}{q}\right)^{\frac1q} \geq 1
\end{equation*}

Following the path in \cite{nicolatilli_norm} we obtain ...\\
\begin{equation*}
	G'(u(t)) = \lambda_1 t^{p-1} + \lambda_2 t^{q-1} \implies u(t) = \dfrac{1}{d!}\left[-\log\left(\lambda_1 t^{p-1} + \lambda_2 t^{q-1}\right)\right]^d, \ t \in (0,M)
\end{equation*}
Our main goal now is to show that multipliers $\lambda_1, \lambda_2$ are unique and both positive.\\
The easiest fact to prove is that both multipliers are not 0. In fact if one, say $\lambda_2$, was 0, we would obtain that the solution of our problem is the same as the one with just the $L^p$ bound. But we already know that this function does not satisfy the $L^q$ constraint hence it is impossible that $\lambda_2=0$.\\
Suppose now that one of the multipliers, say always $\lambda_2$, is negative. Consider an interval $[a-l, a+l]$ contained in $(0,M)$ and the variation
\[ \eta(t) = \begin{cases*}
	-t^{1-p}, &  $t \in (a-l,a)$  \\
	\phantom{-}t^ {1-p}, & $t \in (a,a+l)$
\end{cases*} \]
$\eta$ is an admissible variation because $u$ is strictly positive in $[a-l, a+l]$, $\int_{a-l}^{a+l}t^{p-1}\eta(t)dt=0$, and $\int_{a-l}^{a+l}t^{q-1}\eta(t)dt <0$ if $q<p$ (otherwise we can take $-\eta$ instead of $\eta)$. The directional derivative of $G$ along $\eta$ is
\begin{align*}
	\int_{a-l}^{a+l} G'(u(t))\eta(t)dt &= \int_{a-l}^{a+l} (\lambda_1 t^{p-1} + \lambda_2 t^{q-1})\eta(t)dt = \\
									   &= \lambda_2 \int_{a-l}^{a+l} t^{q-1} \eta(t)dt > 0
\end{align*}
which contradicts the fact that $u$ is a maximizer.\\
{\Large PROOF OF CONTINUITY }\\
Lastly we shall prove that multipliers $\lambda_1, \lambda_2$, and hence maximizer, are unique.

\nocite{*}
\printbibliography


	

\end{document}32
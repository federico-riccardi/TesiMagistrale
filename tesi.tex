\documentclass[corpo=11pt, stile=classica, tipotesi=custom,
greek, evenboxes]{toptesi}
\usepackage[utf8]{inputenc}
\usepackage[T1]{fontenc}
\usepackage{lmodern}

\usepackage{hyperref}
\hypersetup{%
	pdfpagemode={UseOutlines},
	bookmarksopen,
	pdfstartview={FitH},
	colorlinks,
	linkcolor={blue},
	citecolor={blue},
	urlcolor={blue}
}

\usepackage{geometry} %for the margins
\newcommand\fillin[1][4cm]{\makebox[#1]{\dotfill}} %for the dotted line in the frontispiace

\usepackage{dcolumn}
\newcolumntype{d}{D{.}{.}{-1} } %to vetical align numbers in tables, along the decimal dot

\usepackage{amsmath}

\usepackage{natbib} % for the bibliography
\bibliographystyle{plainnat}


\usepackage{amsmath, amssymb, amsthm}
\usepackage{enumitem}
\usepackage{tikz}
\usepackage{hyperref}
\usepackage{systeme}
\usepackage{mathtools}
\hypersetup{colorlinks, linkcolor=blue, citecolor=blue}
\usepackage{natbib} % for the bibliography
\bibliographystyle{plain}


\numberwithin{equation}{chapter}
\newtheorem{teo}{Theorem}[chapter] %in questo modo la numerazione ricomincia da capo ad ogni nuovo capitolo
\newtheorem{defi}[teo]{Definition}
\newtheorem{lem}[teo]{Lemma}
\newtheorem{cor}[teo]{Corollary}
\newtheorem{prop}[teo]{Proposition}
\newtheorem{es}[teo]{Example}

\newcommand{\RR}{\mathbb{R}} %scorciatoia per R
\newcommand{\notazione}{\underline{\textbf{Remark Notazionale}}}
\newcommand{\finire}{\fbox{\LARGE DA FINIRE}}

\begin{document}
	
\begin{titlepage}
\newgeometry{left=1cm,right=1cm,top=3cm,bottom=3.5cm}  %specific margins for this page

\begin{center}

{\huge POLITECNICO DI TORINO}\\[1.5cm]
%\textbf{Corso di Laurea\\in Matematica per l'Ingegneria}\\[3cm]
\textbf{Corso di Laurea Magistrale\\in Ingegneria Matematica}\\[3cm]

%{\Large Tesi di Laurea}\\[1cm]
{\Large Tesi di Laurea Magistrale}\\[0.5cm]
\textbf{\LARGE Some recent results on the norm of localization operators}\\[2cm]
\includegraphics[width=0.2\textwidth]{./Pictures/logo_polito_2021.jpg}
\vspace{4cm}


\begin{minipage}{0.85\textwidth}
\begin{flushleft}\large
\textbf{Relatori} \hfill \textbf{Candidato}\\
Prof. Fabio Nicola \hfill Federico Riccardi\\
\textit{firma dei relatori} \hfill \textit{firma del candidato}\\[0.35cm]
\fillin\ \hfill \fillin
\end{flushleft}
\end{minipage}

\vfill

Anno Accademico 2022-2023
\end{center}

\restoregeometry %restor default margins 

\end{titlepage}

\sommario

\chapter{Introduction}

\chapter{Basics of functional analysis}

\chapter{Short-Time Fourier Transform}
\section{STFT}
\subsection{Properties of STFT}
\section{Fock Space and Bargmann Transform}
\section{Faber-Krahn Inequality for the STFT}


\chapter{Localization Operators}
\section{Definition and properties}
\section{Eigenvalues and eigenfunctions}

\chapter{Recent results from Nicola-Tilli}
\section{Case $q=+\infty$}
\section{Generic case}

\tableofcontents


\nocite{*}
\bibliography{bibliografia.bib}


	

\end{document}32